\documentclass[UTF8,11pt]{article}
\usepackage{graphicx}
\usepackage{indentfirst}
\usepackage{subfigure}
\usepackage[UTF8]{ctex}
\usepackage{amsmath}
\usepackage{enumitem}
\usepackage{geometry}
\usepackage{float}
\usepackage{wrapfig}
\usepackage{setspace}
\usepackage{mathtools}
\usepackage{indentfirst}
\usepackage{enumitem}

\setenumerate[1]{itemsep=0pt,partopsep=0pt,parsep=\parskip,topsep=5pt}
\setitemize[1]{itemsep=0pt,partopsep=0pt,parsep=\parskip,topsep=5pt}
\setdescription{itemsep=0pt,partopsep=0pt,parsep=\parskip,topsep=5pt}

\setlength{\parindent}{0em}
\geometry{a4paper,left=2cm, right=2cm, top=2cm, bottom=2cm}
\linespread{1.4}

\title{千丛恋雨万花绫高考冲刺卷\quad 理科综合}
\author{}
\date{}

\begin{document}
$$\text{\Huge \textbf{千丛恋雨万花绫高考冲刺卷}}$$
$$\text{\huge \textbf{理科综合}}$$


\textbf{一、选择题(每题6分)}

1.中华文化博大精深,源远流长。有许多诗句中暗藏生物学原理,下列说法中正确的是( \qquad )

\begin{itemize}
    \item[A] “数罟不入洿池,鱼鳖不可胜食也”体现了年龄组成能直接影响种群密度
    \item[B] “心非木石岂无感,吞声踯躅不敢言”体现了只有动物可以主动适应外部环境,进行自我调节
    \item[C] “落红不是无情物,化作春泥更护花”体现了分解者具有分解植物遗体为无机盐的功能
    \item[D] “汉川修竹贱如蓬,斤斧何曾赦箨龙”体现了生物多样性的直接和间接价值
\end{itemize}

2.下列有关生物学的传言,一定错误的是(\qquad)

\begin{itemize}
    \item[A] 通过红外检测装置,一位教授发现:在喝一杯热茶后,人体体表温度发生了明显(1℃左右)的变化
    \item[B] 某实验室发现野生大肠杆菌中有一种修复自身损伤DNA末端的酶,这一发现有助于研究癌症
    \item[C] 据某间谍透露,某国试图研究一种病毒生物武器,但是由于缺少活的人体细胞而失败
    \item[D] 某医疗机构发现了一种新冠病毒突变型,该突变型传染性明显低于其他种类的新冠病毒
\end{itemize}

3.下列有关人体信号分子的叙述,正确的是(\qquad)

\begin{itemize}
    \item[A] 青春期时,性激素与靶细胞细胞膜上的受体特异性结合,促进生殖器官发育
    \item[B] 生长素经体液运输到达骨骼,改变骨骼细胞状态
    \item[C] 病原体入侵人体时,T细胞产生的淋巴因子促进浆细胞分裂分化
    \item[D] 经60℃的加热处理后,部分胞吐释放的信息分子仍然可以正常发挥作用
\end{itemize}

4.植物光合作用的光谱是通过测量光合作用对不同波长的光的反应速率(如$O_2$的释放速率)来绘制的。下列叙述错误的是(\qquad)

\begin{itemize}
    \item[A] 类胡萝卜素在红光区吸收的光能可用于光反应中ATP的合成
    \item[B] 叶绿素的吸收光谱可通过测量其对不同波长的光的吸收值来绘制
    \item[C] 光合作用的光谱也可用$CO_2$的吸收速率随光波长的变化表示
    \item[D] 叶片在640~660nm波长光下释放$O_2$是由叶绿素参与光合作用引起的
\end{itemize}

5.已知猫的花色受到X染色体上的一对等位基因控制,且Y染色体上没有相应的等位基因。猫的有尾和无尾受到常染色体上另一对等位基因控制。当花色基因杂合时,表现为玳瑁猫。现有甲、乙、丙、丁四种猫,其中甲、乙尾雌性,丙、丁为雄性。表现型如表所示。已知乙和丁杂交,得到有尾:无尾=1:1;乙和丙杂交,子代全为有尾。取甲、丁杂交后代中的有尾玳瑁猫验证分离定律,则应和(\qquad)杂交

\begin{itemize}
    \item[A] 甲
    \item[B] 乙
    \item[C] 丙
    \item[D] 丁
\end{itemize}

6.线粒体内膜的通透性很差,$H^{+}$和[H]几乎完全不能以自由扩散的方式穿过线粒体内膜。线粒体内膜上有$H^{+}$通道蛋白-ATP酶复合体,复合体在顺浓度梯度运输$H^{+}$的同时,利用释放的能量合成ATP,此外,线粒体内膜上的一种载体蛋白可利用[H]与$O_2$反应生成水时释放的能量主动运输$H^{+}$.已知DNP是一种减肥药物,它可以增加线粒体内膜对$H^{+}$的通透性,使$H^{+}$可以自由扩散穿过线粒体内膜。但DNP有使体温过度升高的副作用,故因为不安全而被禁用。下列说法中正确的是(\qquad)

\begin{itemize}
    \item[A] 图中$H^{+}$浓度高的一侧为线粒体基质
    \item[B] DNP主要作用于下丘脑,下丘脑调节体温升高
    \item[C] DNP使得有氧呼吸第三阶段产生的ATP减少
    \item[D] DNP通过提高有氧呼吸效率起到减肥效果
\end{itemize}

7.

\end{document}