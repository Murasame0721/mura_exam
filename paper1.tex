\documentclass[UTF8,11pt]{article}
\usepackage{graphicx}
\usepackage{indentfirst}
\usepackage{subfigure}
\usepackage[UTF8]{ctex}
\usepackage{amsmath}
\usepackage{enumitem}
\usepackage{geometry}
\usepackage{float}
\usepackage{wrapfig}
\usepackage{setspace}
\usepackage{mathtools}
\usepackage{indentfirst}
\usepackage{enumitem}

\setenumerate[1]{itemsep=0pt,partopsep=0pt,parsep=\parskip,topsep=5pt}
\setitemize[1]{itemsep=0pt,partopsep=0pt,parsep=\parskip,topsep=5pt}
\setdescription{itemsep=0pt,partopsep=0pt,parsep=\parskip,topsep=5pt}

\setlength{\parindent}{0em}
\geometry{a4paper,left=2cm, right=2cm, top=2cm, bottom=2cm}
\linespread{1.4}

\title{千丛恋雨万花绫高考冲刺卷\quad 理科综合}
\author{}
\date{}

\begin{document}
$$\text{\Huge \textbf{千丛恋雨万花绫高考冲刺卷}}$$
$$\text{\huge \textbf{理科综合}}$$


\textbf{一、选择题(每题6分)}

1.中华文化博大精深,源远流长。有许多诗句中暗藏生物学原理,下列说法中正确的是( \qquad )

\begin{itemize}
    \item[A] “数罟不入洿池,鱼鳖不可胜食也”体现了年龄组成能直接影响种群密度
    \item[B] “心非木石岂无感,吞声踯躅不敢言”体现了只有动物可以主动适应外部环境,进行自我调节
    \item[C] “落红不是无情物,化作春泥更护花”体现了分解者具有分解植物遗体为无机盐的功能
    \item[D] “汉川修竹贱如蓬,斤斧何曾赦箨龙”体现了生物多样性的直接和间接价值
\end{itemize}

2.下列有关生物学的传言,一定错误的是(\qquad)

\begin{itemize}
    \item[A] 通过红外检测装置,一位教授发现:在喝一杯热茶后,人体体表温度发生了明显(1℃左右)的变化
    \item[B] 某实验室发现野生大肠杆菌中有一种修复自身损伤DNA末端的酶,这一发现有助于研究癌症
    \item[C] 据某间谍透露,某国试图研究一种病毒生物武器,但是由于缺少活的人体细胞而失败
    \item[D] 某医疗机构发现了一种新冠病毒突变型,该突变型传染性明显低于其他种类的新冠病毒
\end{itemize}

3.下列有关人体信号分子的叙述,正确的是(\qquad)

\begin{itemize}
    \item[A] 青春期时,性激素与靶细胞细胞膜上的受体特异性结合,促进生殖器官发育
    \item[B] 生长素经体液运输到达骨骼,改变骨骼细胞状态
    \item[C] 病原体入侵人体时,T细胞产生的淋巴因子促进浆细胞分裂分化
    \item[D] 经60℃的加热处理后,部分胞吐释放的信息分子仍然可以正常发挥作用
\end{itemize}

4.植物光合作用的光谱是通过测量光合作用对不同波长的光的反应速率(如$O_2$的释放速率)来绘制的。下列叙述错误的是(\qquad)

\begin{itemize}
    \item[A] 类胡萝卜素在红光区吸收的光能可用于光反应中ATP的合成
    \item[B] 叶绿素的吸收光谱可通过测量其对不同波长的光的吸收值来绘制
    \item[C] 光合作用的光谱也可用$CO_2$的吸收速率随光波长的变化表示
    \item[D] 叶片在640~660nm波长光下释放$O_2$是由叶绿素参与光合作用引起的
\end{itemize}

5.已知猫的花色受到X染色体上的一对等位基因控制,且Y染色体上没有相应的等位基因。猫的有尾和无尾受到常染色体上另一对等位基因控制。当花色基因杂合时,表现为玳瑁猫。现有甲、乙、丙、丁四种猫,其中甲、乙尾雌性,丙、丁为雄性。表现型如表所示。已知乙和丁杂交,得到有尾:无尾=1:1;乙和丙杂交,子代全为有尾。取甲、丁杂交后代中的有尾玳瑁猫验证分离定律,则应和(\qquad)杂交

\begin{itemize}
    \item[A] 甲
    \item[B] 乙
    \item[C] 丙
    \item[D] 丁
\end{itemize}

6.线粒体内膜的通透性很差,$H^{+}$和[H]几乎完全不能以自由扩散的方式穿过线粒体内膜。线粒体内膜上有$H^{+}$通道蛋白-ATP酶复合体,复合体在顺浓度梯度运输$H^{+}$的同时,利用释放的能量合成ATP,此外,线粒体内膜上的一种载体蛋白可利用[H]与$O_2$反应生成水时释放的能量主动运输$H^{+}$.已知DNP是一种减肥药物,它可以增加线粒体内膜对$H^{+}$的通透性,使$H^{+}$可以自由扩散穿过线粒体内膜。但DNP有使体温过度升高的副作用,故因为不安全而被禁用。下列说法中正确的是(\qquad)

\begin{itemize}
    \item[A] 图中$H^{+}$浓度高的一侧为线粒体基质
    \item[B] DNP主要作用于下丘脑,下丘脑调节体温升高
    \item[C] DNP使得有氧呼吸第三阶段产生的ATP减少
    \item[D] DNP通过提高有氧呼吸效率起到减肥效果
\end{itemize}

7.生活中的化学产品与我们的健康息息相关,下列说法中正确的是(\qquad)
\begin{itemize}
    \item[A] 维生素AD片(成分为烃,醇、脂质,淀粉)属于非电解质
    \item[B] 相较于带鱼,食用柠檬、山楂更易导致体内酸性物质积累
    \item[C] 使用奶油和人造奶油时,无需注意贮藏方式以防
    \item[D] 在制作玻璃时,加入金属离子可制成钢化玻璃,改善玻璃性能
\end{itemize}

8.腺嘌呤核糖核苷酸(下图)是生物体内一种重要的物质,是RNA合成原料之一。嘌呤核糖核苷酸(以下简称分子A)由腺嘌呤、核糖、磷酸基基团三部分组成,其中腺嘌呤部分含有苯类似的结构,核糖与磷酸基团通过酯基相连,则下列说法中错误的是(\qquad)
\begin{itemize}
 \item[A] 分子所有N原子一定共平面
 \item[B]A分子的分子式为${{\text{C}}_{\text{10}}}{{\text{H}}_{\text{15}}}{{\text{N}}_{\text{5}}}{{\text{O}}_{\text{2}}}\text{P}$
\item[C]A分子可发生取代反应、氧化反应
 \item[D]A分子只有4个性碳(与碳原子相连的4个基因各不相同)
\end{itemize}

9.透明溶液在低价酚酞溶液后颜色最终无变化,则溶液中可能只大量含有(\qquad)
\begin{itemize}
 \item[A] $\text{Cl}{{\text{O}}^{-}}$,${{\text{K}}^{+}}$,$\text{C}{{\text{l}}^{-}}$					 \item[B]$\text{M}{{\text{g}}^{2+}}$,$\text{B}{{\text{a}}^{2+}}$,$\text{So}_{4}^{2-}$
\item[C]$\text{N}{{\text{a}}^{+}}$,$\text{SO}_{4}^{2-}$,$\text{CO}_{3}^{2-}$					 \item[D]$\text{AlO}_{2}^{-}$,$\text{N}{{\text{a}}^{+}}$,$\text{C}{{\text{l}}^{-}}$
\end{itemize}

10.短周期元素X、Y、Z、W原子序数依次增大,位于四个不同的主族。X的简单氢化物与W单质按物质的两1:2混合反应,得5种生成物,其中只有一种溶于Y的简单氢化物且不发生反应。Y、Z元素可形成如图所示的化合物,则下列说法中正确的是(\qquad)
\begin{itemize}
 \item[A] Z的最高价氧化物的水化物为强酸
 \item[B]W的最低氧化物的水化物有强氧化性
\item[C]Y的简单氢化物的沸点高于X的氢化物
 \item[D]Y和Z是空气的主要元素
\end{itemize}

11.具有强自发倾向的非氧化还原反应也能制成原电池,但电压较低。如图所示,沉淀原电池就是非氧化还原反应的原电池中的一种。下列说法中错误的是(\qquad)
\begin{itemize}
 \item[A] 原电池的总反应方程式$\text{C}{{\text{u}}^{2+}}+2\text{O}{{\text{H}}^{-}}=\text{Cu}{{\left( \text{OH} \right)}_{2}}\downarrow $
 \item[B]盐桥中的离子可以为${{\text{K}}^{+}}$和$\text{C}{{\text{l}}^{-}}$
\item[C]接通后,电流由电流表左端流出,右端流入
 \item[D]将KOH溶液替换为$\text{N}{{\text{a}}_{2}}\text{S}$溶液,电流方向不改变
\end{itemize}

12.某化学兴趣小组利用${{\text{K}}_{\text{2}}}\text{C}{{\text{r}}_{\text{2}}}{{\text{O}}_{\text{7}}}$和${{\text{H}}_{\text{2}}}{{\text{O}}_{\text{2}}}$的特殊非氧化还原反应制备$\text{Cr}{{\text{O}}_{\text{5}}}$,并萃取、保存,实验装置如下图。已知:${{\text{K}}_{\text{2}}}\text{C}{{\text{r}}_{\text{2}}}{{\text{O}}_{\text{7}}}$不能将${{\text{H}}_{\text{2}}}{{\text{O}}_{\text{2}}}$氧化为${{\text{O}}_{2}}$。$\text{Cr}{{\text{O}}_{\text{5}}}$在水中分解,酸性环境下分解加快,但不再乙醚中分解。$\text{Cr}{{\text{O}}_{\text{5}}}$的制备方程式:${{\text{K}}_{\text{2}}}\text{C}{{\text{r}}_{\text{2}}}{{\text{O}}_{\text{7}}}+4{{\text{H}}_{\text{2}}}{{\text{O}}_{\text{2}}}+{{\text{H}}_{\text{2}}}\text{S}{{\text{O}}_{\text{4}}}$(稀)$=2\text{Cr}{{\text{O}}_{5}}+2{{\text{H}}_{2}}\text{O}+{{\text{K}}_{2}}\text{S}{{\text{O}}_{4}}$.下列说法中错误的是(\qquad)
\begin{itemize}
 \item[A] a处${{\text{N}}_{2}}$可换为空气,且只需在d中反应结束后通入
 \item[B]b液体可为${{\text{H}}_{\text{2}}}{{\text{O}}_{\text{2}}}$溶液或${{\text{H}}_{\text{2}}}{{\text{O}}_{\text{2}}}$,${{\text{K}}_{\text{2}}}\text{C}{{\text{r}}_{\text{2}}}{{\text{O}}_{\text{7}}}$混合溶液
\item[C]将液体推入e时,应保持c打开
 \item[D]$\text{Cr}{{\text{O}}_{\text{5}}}$分解离子方程$4\text{Cr}{{\text{O}}_{\text{5}}}+12{{\text{H}}^{+}}=4\text{C}{{\text{r}}^{3+}}+6{{\text{H}}_{2}}\text{O}+7{{\text{O}}_{2}}\uparrow $
\end{itemize}

13.${{\text{K}}_{\text{2}}}\text{C}{{\text{r}}_{\text{2}}}{{\text{O}}_{\text{7}}}$溶液中存在平衡$\text{C}{{\text{r}}_{\text{2}}}\text{O}_{7}^{2-}+{{\text{H}}_{2}}\text{O}2\text{CrO}_{4}^{2-}+2{{\text{H}}^{+}}$。现将$\text{0}\text{.1mol}{{\text{K}}_{\text{2}}}\text{C}{{\text{r}}_{\text{2}}}{{\text{O}}_{\text{7}}}$配成1L溶液,保温至25℃(忽略溶液体积变化)。由实验测得$\lg \frac{{{\text{C}}^{2}}\left( \text{CrO}_{4}^{2-} \right)}{\text{C}\left( \text{C}{{\text{r}}_{\text{2}}}\text{O}_{7}^{2-} \right)}$具有如图线性关系,则下列说法中错误的是(\qquad)(参考数据:$\lg \left( 5\sqrt{10}+1 \right)=1.226$,$\frac{\sqrt{65}}{20}=0.40$)
\begin{itemize}
 \item[A] 反应$\text{CrO}_{4}^{2-}+2{{\text{H}}^{+}}\text{C}{{\text{r}}_{\text{2}}}\text{O}_{7}^{2-}+{{\text{H}}_{2}}\text{O}$的平衡常数为${{10}^{14}}$
 \item[B]PH为1时,$\text{C}\left( \text{C}{{\text{r}}_{\text{2}}}\text{O}_{7}^{2-} \right)+\text{C}\left( {{\text{H}}^{+}} \right)<0.2\text{mol/L}$
\item[C]PH为7时,$\text{C}\left( \text{CrO}_{4}^{2-} \right)=0.5\text{mol/L}$
 \item[D]PH为5.774时,$\text{C}\left( {{\text{H}}^{+}} \right)<\text{C}\left( \text{CrO}_{4}^{2-} \right)+<\left( \text{O}{{\text{H}}^{-}} \right)$
(应注明使用NaOH固体和浓${{\text{H}}_{\text{2}}}\text{S}{{\text{O}}_{\text{4}}}$)
物理选择(14-18单选,19-21多选)
\end{itemize}

14.(2016全国Ⅱ)质量为m的物体用轻绳$AB$悬挂于天花板上,用水平向左的力$F$缓慢拉动绳的中点$O$,如图所示,用$T$表示绳$OA$段拉力的大小,在$O$点,向左移动的过程中(\qquad)
\begin{itemize}
 \item[A] $F$逐渐变大,$T$逐渐变大			 
 \item[B] $F$逐渐变大,$T$逐渐变小
\item[C]$F$逐渐变小,$T$逐渐变大				
 \item[D]$F$逐渐变小,$T$逐渐变小
\end{itemize}

15.曲柄滑道机构如图所示,OA的A端与滑块绞连接,只可转动,$\angle DBC=120{}^\circ $,OA长为r,以角速度$\omega $匀速转动,整个机构没有摩擦,则$\varphi =120{}^\circ $时,C点速度为(\qquad)
\begin{itemize}
 \item[A] 0					
  \item[B]$\frac{\omega r}{2}$				
\item[C]$\frac{\sqrt{3}}{2}\omega r$				
  \item[D]$\omega r$
(应该题号为17)(图为《理论力学I》P201)
\end{itemize}

16.如图所示电路中,${{k}_{1}}{{k}_{2}}$之间接一智能电源用电流传感器电路中电流,线圈L的电感很大,电阻未知,电容C的击穿电压足够大,电容很大,则下列情况中不可能发生的是(\qquad)
\begin{itemize}
 \item[A] 智能电源接入交流电时,灯泡不亮,电流传感器示数约为0
 \item[B]智能电源接入交流电时,灯泡亮度恒定,电流传感器示数变化很快
 \item[C]智能电源接入直流电时,灯泡先亮后熄灭,电流传感器示数不断上升
 \item[D]智能电源接入直流电时,灯泡亮度逐渐降低后稳定,电流传感器示数先上升后不变
\end{itemize}


\end{document}
