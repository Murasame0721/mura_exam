\documentclass[UTF8,11pt]{article}
\usepackage{graphicx}
\usepackage{indentfirst}
\usepackage{subfigure}
\usepackage[UTF8]{ctex}
\usepackage{amsmath}
\usepackage{enumitem}
\usepackage{geometry}
\usepackage{float}
\usepackage{wrapfig}
\usepackage{setspace}
\usepackage{mathtools}
\usepackage{indentfirst}
\usepackage{enumitem}
\usepackage{ulem}
\usepackage{relsize}

\setenumerate[1]{itemsep=0pt,partopsep=0pt,parsep=\parskip,topsep=5pt}
\setitemize[1]{itemsep=0pt,partopsep=0pt,parsep=\parskip,topsep=5pt}
\setdescription{itemsep=0pt,partopsep=0pt,parsep=\parskip,topsep=5pt}

\setlength{\parindent}{0em}
\geometry{a4paper,left=1cm, right=1cm, top=1.5cm, bottom=1.5cm}
\linespread{1.4}

\title{千丛恋雨万花绫高考冲刺卷\quad 理科综合}
\author{}
\date{}

\begin{document}

\begin{figure}[H]
    \includegraphics{head@4x.png}
\end{figure}


\textbf{一、选择题:本题共13小题,每小题6分。在每小题给出的四个选项中,只有一个是符合题目要求的}

1.中华文化博大精深,源远流长。有许多诗句中暗藏生物学原理,下列说法中正确的是( \qquad )

\begin{itemize}
    \item[A] “数罟不入洿池,鱼鳖不可胜食也”体现了年龄组成能直接影响种群密度
    \item[B] “心非木石岂无感,吞声踯躅不敢言”体现了只有动物可以主动适应外部环境,进行自我调节
    \item[C] “落红不是无情物,化作春泥更护花”体现了分解者具有分解遗体,从而促进能量循环利用的功能
    \item[D] “汉川修竹贱如蓬,斤斧何曾赦箨龙”体现了生物多样性的直接价值
\end{itemize}

2.下列有关生物学的传言,一定错误的是(\qquad)

\begin{itemize}
    \item[A] 通过红外检测装置,一位教授发现:在喝一杯热茶后,人体体表温度发生了明显(1℃左右)的变化
    \item[B] 某实验室发现野生大肠杆菌中有一种修复自身损伤DNA末端的酶,这一发现有助于研究癌症
    \item[C] 据某间谍透露,某国试图研究一种病毒生物武器,但是由于缺少活的人体细胞而失败
    \item[D] 某医疗机构发现了一种新冠病毒突变型,该突变型传染性明显低于其他种类的新冠病毒
\end{itemize}

3.下列有关人体信号分子的叙述,正确的是(\qquad)

\begin{itemize}
    \item[A] 口渴时,垂体分泌的抗利尿激素增加,尿量减少
    \item[B] 生长素经体液运输到达骨骼,改变骨骼细胞状态
    \item[C] 病原体入侵人体时,T细胞产生的淋巴因子促进浆细胞分裂分化
    \item[D] 经60℃的加热处理后,部分胞吐释放的信息分子仍然可以正常发挥作用
\end{itemize}

4.植物光合作用的光谱是通过测量光合作用对不同波长的光的反应速率(如$\rm O_2$的释放速率)来绘制的。下列叙述错误的是(\qquad)

\begin{itemize}
    \item[A] 类胡萝卜素在红光区吸收的光能可用于光反应中ATP的合成
    \item[B] 叶绿素的吸收光谱可通过测量其对不同波长的光的吸收值来绘制
    \item[C] 光合作用的光谱也可用$\rm CO_2$的吸收速率随光波长的变化表示
    \item[D] 叶片在640-660nm波长光下释放$\rm O_2$是由叶绿素参与光合作用引起的
\end{itemize}

5.已知猫的花色受到X染色体上的一对等位基因控制,且Y染色体上没有相应的等位基因。猫的有尾和无尾受到常染色体上另一对等位基因控制。当花色基因杂合时,表现为玳瑁猫。现有甲、乙、丙、丁四种猫,其中甲、乙为雌性,丙、丁为雄性。表现型如表所示。已知乙和丁杂交,得到有尾:无尾=1:1;乙和丙杂交,子代全为有尾。取甲、丁杂交后代中的有尾玳瑁猫验证分离定律,则应和(\qquad)杂交

\begin{center}
    \begin{tabular}{|c|c|c|c|}
        \hline
        甲   & 乙   & 丙   & 丁   \\
        \hline
        有尾 & 有尾 & 有尾 & 无尾 \\
        \hline
    \end{tabular}
\end{center}

\begin{itemize}
    \item[A] 甲
    \item[B] 乙
    \item[C] 丙
    \item[D] 丁
\end{itemize}

6.线粒体内膜的通透性很差,$\rm H^{+}$和[H]几乎完全不能以自由扩散的方式穿过线粒体内膜。线粒体内膜上有$\rm H^{+}$通道蛋白-ATP酶复合体,复合体在顺浓度梯度运输$\rm H^{+}$的同时,利用释放的能量合成ATP,此外,线粒体内膜上的一种载体蛋白可利用[H]与$\rm O_2$反应生成水时释放的能量主动运输$\rm H^{+}$.已知DNP是一种减肥药物,它可以增加线粒体内膜对$\rm H^{+}$的通透性,使部分$\rm H^{+}$可以自由扩散穿过线粒体内膜。但DNP有使体温过度升高的副作用,故因为不安全而被禁用。下列说法中正确的是(\qquad)

\begin{figure}[H]
    \centering
    \includegraphics[width=0.4\textwidth]{6@4x.png}
\end{figure}

\begin{itemize}
    \item[A] 图中$\rm H^{+}$浓度高的一侧为线粒体基质
    \item[B] DNP主要作用于下丘脑,使下丘脑调节体温升高
    \item[C] DNP使得有氧呼吸第三阶段产生的ATP减少
    \item[D] DNP通过提高有氧呼吸效率起到减肥效果
\end{itemize}

7.生活中的化学产品与我们的健康息息相关,下列说法中正确的是(\qquad)
\begin{itemize}
    \item[A] 维生素AD片(成分为烃、醇、脂质、淀粉)属于非电解质
    \item[B] 相较于带鱼,食用柠檬、山楂更易导致体内酸性物质积累
    \item[C] 使用奶油和人造奶油时,无需特别注意贮藏方式以防止酸败
    \item[D] 在制作玻璃时,加入金属离子可制成钢化玻璃,改善玻璃性能
\end{itemize}

8.具有强自发倾向的非氧化还原反应也能制成原电池,但电压较低。如图所示的沉淀原电池是非氧化还原反应的原电池中的一种。下列说法中错误的是(\qquad)

%缺图:反应装置

\begin{itemize}
    \item[A] 原电池的总反应方程式$\text{C}{{\text{u}}^{2+}}+2\text{O}{{\text{H}}^{-}}=\text{Cu}{{\left( \text{OH} \right)}_{2}}\downarrow $
        \item[B]盐桥中的离子可以为${{\text{K}}^{+}}$和$\text{C}{{\text{l}}^{-}}$
        \item[C]电路接通后,电流由电流表左端流出,右端流入
        \item[D]将KOH溶液替换为$\text{N}{{\text{a}}_{2}}\text{S}$溶液,电流方向不改变
\end{itemize}

9.腺嘌呤核糖核苷酸(简称AMP,结构如下图)是生物体内一种重要的物质,是RNA合成原料之一。AMP由腺嘌呤、核糖、磷酸三部分组成,其中腺嘌呤部分含有苯类似的结构,核糖与磷酸基团通过酯基相连,则下列说法中错误的是(\qquad)

\begin{figure}[H]
    \centering
    \includegraphics[width=0.3\textwidth]{AMP.png}
    \label{fig:AMP}
\end{figure}

\begin{itemize}
    \item[A] AMP分子所有N原子一定共平面
    \item[B] AMP分子的分子式为${{\text{C}}_{\text{10}}}{{\text{H}}_{\text{15}}}{{\text{N}}_{\text{5}}}{{\text{O}}_{\text{7}}}\text{P}$
    \item[C] AMP分子可发生水解反应、氧化反应
    \item[D] AMP分子只有4个手性碳(与碳原子相连的4个基因各不相同时,这个碳原子是手性碳)
\end{itemize}

10.透明溶液在滴加酚酞溶液后颜色最终无变化,则溶液中可能只大量含有(\qquad)
\begin{itemize}
    \item[A] $\text{Cl}{{\text{O}}^{-}}$,${{\text{K}}^{+}}$,$\text{C}{{\text{l}}^{-}}$			 \item[B]$\text{M}{{\text{g}}^{2+}}$,$\text{B}{{\text{a}}^{2+}}$,$\text{SO}_{4}^{2-}$
        \item[C]$\text{N}{{\text{a}}^{+}}$,$\text{SO}_{4}^{2-}$,$\text{CO}_{3}^{2-}$					 \item[D]$\text{AlO}_{2}^{-}$,$\text{N}{{\text{a}}^{+}}$,$\text{C}{{\text{l}}^{-}}$
\end{itemize}

11.短周期元素X、Y、Z、W原子序数依次增大,位于四个不同的主族。X的简单氢化物与W单质按物质的量1:2混合反应,得5种生成物,其中只有一种生成物溶于Y的简单氢化物且不与其发生反应。Y、Z元素可形成如图所示的化合物,则下列说法中正确的是(\qquad)

\begin{figure}[H]
    \centering
    \includegraphics[width=0.25\textwidth]{p4o6.png}
    \label{fig:p4o6}
\end{figure}

\begin{itemize}
    \item[A] Z的最高价氧化物的水化物为强酸
    \item[B] W的最低氧化物的水化物有强氧化性
    \item[C] Y的简单氢化物的沸点高于X的氢化物
    \item[D] Y和Z是空气的主要元素
\end{itemize}

12.某化学兴趣小组利用${{\text{K}}_{\text{2}}}\text{C}{{\text{r}}_{\text{2}}}{{\text{O}}_{\text{7}}}$和${{\text{H}}_{\text{2}}}{{\text{O}}_{\text{2}}}$的特殊非氧化还原反应制备$\text{Cr}{{\text{O}}_{\text{5}}}$,并萃取、保存,实验装置如下图。已知:${{\text{K}}_{\text{2}}}\text{C}{{\text{r}}_{\text{2}}}{{\text{O}}_{\text{7}}}$不能将${{\text{H}}_{\text{2}}}{{\text{O}}_{\text{2}}}$氧化为${{\text{O}}_{2}}$。$\text{Cr}{{\text{O}}_{\text{5}}}$在水中分解,酸性环境下分解加快,但不在乙醚中分解。$\text{Cr}{{\text{O}}_{\text{5}}}$的制备方程式:${{\text{K}}_{\text{2}}}\text{C}{{\text{r}}_{\text{2}}}{{\text{O}}_{\text{7}}}+4{{\text{H}}_{\text{2}}}{{\text{O}}_{\text{2}}}+{{\text{H}}_{\text{2}}}\text{S}{{\text{O}}_{\text{4}}}$(稀)$=2\text{Cr}{{\text{O}}_{5}}+2{{\text{H}}_{2}}\text{O}+{{\text{K}}_{2}}\text{S}{{\text{O}}_{4}}$.下列说法中错误的是(\qquad)

%缺图:反应装置

\begin{itemize}
    \item[A] a处${{\text{N}}_{2}}$可换为空气,且只需在d中反应结束后通入
    \item[B] b液体可为${{\text{H}}_{\text{2}}}{{\text{O}}_{\text{2}}}$溶液或${{\text{H}}_{\text{2}}}{{\text{O}}_{\text{2}}}$、${{\text{K}}_{\text{2}}}\text{C}{{\text{r}}_{\text{2}}}{{\text{O}}_{\text{7}}}$混合溶液
    \item[C] 将液体推入e时,应保持c打开
    \item[D] $\text{Cr}{{\text{O}}_{\text{5}}}$分解离子方程$4\text{Cr}{{\text{O}}_{\text{5}}}+12{{\text{H}}^{+}}=4\text{C}{{\text{r}}^{3+}}+6{{\text{H}}_{2}}\text{O}+7{{\text{O}}_{2}}\uparrow $
\end{itemize}

13.${{\text{K}}_{\text{2}}}\text{C}{{\text{r}}_{\text{2}}}{{\text{O}}_{\text{7}}}$溶液中存在平衡$\text{C}{{\text{r}}_{\text{2}}}\text{O}_{7}^{2-}+{{\text{H}}_{2}}\text{O}2\text{CrO}_{4}^{2-}+2{{\text{H}}^{+}}$。现用0.1mol/L的$\text{0}\text{.1mol}{{\text{K}}_{\text{2}}}\text{C}{{\text{r}}_{\text{2}}}{{\text{O}}_{\text{7}}}$溶液、$\rm NaOH$固体、浓硫酸研究这一关系。将溶液保温至25℃(忽略溶液体积变化),由实验测得$\displaystyle  \lg \frac{{{\text{C}}^{2}}\left( \text{CrO}_{4}^{2-} \right)}{\text{C}\left( \text{C}{{\text{r}}_{\text{2}}}\text{O}_{7}^{2-} \right)}$具有如图线性关系,则下列说法中错误的是(\qquad)(参考数据:$\lg \left( 5\sqrt{10}+1 \right)=1.226$,$\displaystyle \frac{\sqrt{65}}{20}=0.40$)

%缺图:直线

\begin{itemize}
    \item[A] 反应$\text{CrO}_{4}^{2-}+2{{\text{H}}^{+}}\text{C}{{\text{r}}_{\text{2}}}\text{O}_{7}^{2-}+{{\text{H}}_{2}}\text{O}$的平衡常数为${{10}^{14}}$
    \item[B] pH为1时,$\text{C}\left( \text{C}{{\text{r}}_{\text{2}}}\text{O}_{7}^{2-} \right)+\text{C}\left( {{\text{H}}^{+}} \right)<0.2\text{mol/L}$
    \item[C] pH为7时,$\text{C}\left( \text{CrO}_{4}^{2-} \right)=0.5\text{mol/L}$
    \item[D] pH为5.774时,$\text{C}\left( {{\text{H}}^{+}} \right)<\text{C}\left( \text{CrO}_{4}^{2-} \right)+<\left( \text{O}{{\text{H}}^{-}} \right)$
\end{itemize}

\textbf{二、选择题:本题共8小题,每小题6分。在每小题给出的四个选项中,第14-18题只有一个选项正确,第19-21题有多个选项正确。全部选对的得6分,选对但不全的得3分,有选错的得0分}

14.质量为m的物体用轻绳$AB$悬挂于天花板上,用水平向左的力$F$缓慢拉动绳的中点$O$,如图所示,用$T$表示绳$OA$段拉力的大小,在$O$点向左移动的过程中(\qquad)

\begin{figure}[H]
    \centering
    \includegraphics[width=0.15\textwidth]{14@4x.png}
\end{figure}

\begin{itemize}
    \item[A] $F$逐渐变大,$T$逐渐变大
    \item[B] $F$逐渐变大,$T$逐渐变小
    \item[C] $F$逐渐变小,$T$逐渐变大
    \item[D] $F$逐渐变小,$T$逐渐变小
\end{itemize}

15.如图所示电路中,${{K}_{1}}{{K}_{2}}$之间接一智能电源,用电流传感器(图中表示为电流表)测电路中电流。线圈L的电感很大,电阻未知;电容C的击穿电压足够大,电容很大。则下列情况中不可能发生的是(\qquad)

\begin{figure}[H]
    \centering
    \includegraphics[width=0.25\textwidth]{16@4x.png}
\end{figure}

\begin{itemize}
    \item[A] 智能电源接入交流电时,灯泡不亮,电流传感器示数约为0
        \item[B]智能电源接入交流电时,灯泡亮度恒定,电流传感器示数变化很快
        \item[C]智能电源接入直流电时,灯泡先亮后熄灭,电流传感器示数不断上升
        \item[D]智能电源接入直流电时,灯泡亮度逐渐降低后稳定,电流传感器示数先上升后不变
\end{itemize}

16.2020年12月4日14时2分,新一代“人造太阳”装置——中国环流器二号(HL-2M)在成都建成并首次放电。装置内核反应方程式为$\rm _1^2 H +_2^1 H \rightarrow _2^3 He + _0^1 n$.已知$_1^2 H$的质量为2.0316u,$_2^3 He$的质量为3.0150u,$_0^1 n$
的质量为1.0087u. $\rm 1u= 931MeV/c^2$,则该反应释放的核能约为(\qquad)
\begin{itemize}
    \item[A] 3.7MeV
    \item[B] 3.3MeV
    \item[C] 2.7MeV
    \item[D] 0.93MeV
\end{itemize}

17.曲柄滑道机构如图所示,OA的A端与滑块绞连接,只可转动,$\angle DBC=150{}^\circ $,OA长为r,以角速度$\omega $逆时针匀速转动,整个机构没有摩擦,则$\varphi =150{}^\circ $时,C点速度为(\qquad)

\begin{figure}[H]
    \centering
    \includegraphics[width=0.4\textwidth]{15@4x.png}
\end{figure}

\begin{itemize}
    \item[A] 0
        \item[B]$\displaystyle \frac{\omega r}{2}$
        \item[C]$\displaystyle \frac{\sqrt{3}}{2}\omega r$
        \item[D]$\omega r$
\end{itemize}

18.如图1,抛物线$y=x^2$上方有垂直纸面向内的匀强磁场,磁感应强度为B。抛物线下方有沿x轴负方向,大小为$\displaystyle E=\frac{B^2 q}{8m}$的匀强电场。在电场中释放一些电荷量为q(q>0),质量为m的粒子。粒子初速度$v_0$方向平行于x轴。如图2,直线$x=a$、$x=b$、$y=x^2$、 $y=0$ (a<b)围成的图形的面积为$\displaystyle \frac{b^3 -a^3}{3}$,则下列叙述中错误的是(\qquad)

%缺图:复杂坐标系

\begin{itemize}
    \item[A] 无论在什么位置释放粒子,也无论$v_0$取什么值,都不可能使得粒子从磁场垂直x轴穿过原点
    \item[B] 若$v_0=0$,在x轴负半轴上$-1<x<1$的区域释放粒子,粒子扫过的面积为$\displaystyle \frac{4}{3}+ \pi $
    \item[C] 若在x轴正半轴任意一点上以$\displaystyle v_0 = \frac{Bq}{4m}$释放粒子,则粒子不会进入到磁场区域
    \item[D] 若将E改为$\frac{B^2 q}{2m}$,在x轴负半轴任意一点释放粒子,都会使粒子第一次进入磁场的点恰好和第一次离开磁场的点关于y轴对称
\end{itemize}


19.如图,货运飞船在低轨道I的A点启动发动机,经过椭圆转移轨道II的一半后,在高轨道III的B点再次启动发动机,与III上的空间站对接。轨道I、II、III的半径分别为$r_1$、$r_2$、$r_3$,周期分别为$T_1$、$T_2$、$T_3$,则下列叙述中正确的是(\qquad)

\begin{figure}[H]
    \centering
    \includegraphics[width=0.25\textwidth]{19@4x.png}
\end{figure}

\begin{itemize}
    \item[A] $\displaystyle \sqrt[3]{\frac{T_1 T_2}{T_2^2}} = \frac{8\sqrt{r_1 r_2}}{r_1 +r_2}$
    \item[B] 在轨道II上由A点向B点移动时,发动机推力始终克服万有引力做正功
    \item[C] 若货运飞船启动发动机时空间站在C点,则一定有$\displaystyle \angle COB= \frac{\pi}{\sqrt{2}} \left( \frac{r_1+r_2}{r_2} \right)^{\frac{3}{2}}$
    \item[D] 飞船在轨道III的B点时的机械能大于在轨道II上任意一点时的机械能
\end{itemize}

20.由于地球的自转,在以地球为参考系时,物体并不完全遵循牛顿运动定律。尽管地球的自转对物体的影响 很小,但在研究河流运动、炮弹弹道时,这种影响不可忽略。为使牛顿运动定律在形式上成立,假想存在地转偏向力,大小为$F_* =2m \omega v \sin \theta$,其中$m$为物体的质量,$\omega$为地球自转角速度,$v$为物体相对于地面的速度,$\theta$为物体相对速度与地轴的夹角。$F_*$的方向遵循左手定则。如图,由于地转偏向力,北半球副热带高气压带与赤道之间本应持续向正南的风被偏转向西南,形成东北信风带。下列叙述中正确的是(\qquad)

\begin{figure}[H]
    \centering
    \includegraphics[width=0.4\textwidth]{20@4x.png}
\end{figure}

\begin{itemize}
    \item[A] 由于地转偏向力,北半球自西向东的河流会明显冲刷北岸
    \item[B] 把手张开,令地轴自南向北进入手心,四指指向$v$,则大拇指方向为$F_*$方向
    \item[C] 在南纬$30^\circ$处向正南方与地面夹角$60^\circ$的方向发射质量为$m_0$,速度为$v_0$的炮弹,炮弹所受地转偏向力为$m_0 \omega v_0$
    \item[D] 在位于赤道的光滑水平面上,给小球以$v_1$的初速度($v_1$向东且$\displaystyle \frac{v_1}{2\omega}$远小于地球半径),则小球可以做匀速圆周运动
\end{itemize}

21.对于迎风面积不太大,速度不太大的物体,其所受空气阻力满足$F_f = -kv$,$k$为常数,$v$为物体相对于空气的速度。现以初速度$v_0$竖直上抛质量为$m$的小球,小球升高$h$后落下,整个过程始终满足$F_f = -kv$.下列叙述中正确的是(\qquad)

\begin{itemize}
    \item[A] 经过$\displaystyle \frac{mv_0 -kh}{mg}$时间时,小球到达最高点
    \item[B] 若没有空气阻力,小球会更快回到原位置
    \item[C] 若到达最高点时间与从最高点返回时间比为1:2,则返回时速度为$2v_0 -\frac{kh}{m}$
    \item[D] 重力的最大功率为$\displaystyle \frac{m^2 g^2}{k}$
\end{itemize}

\textbf{三、非选择题:共174分。第22-32题为必考题,每个试题考生必须作答。第33-38题为选考题,考生根据要求作答。}

\textbf{(一)必考题:共129分}

\textbf{22.}某同学利用牛顿第二定律测量气垫导轨滑块的质量。所用器材有:水平桌面上的气垫导轨、上方安装有遮光片的滑块(遮光片质量可忽略)、数字计时器、棉线、安装在气垫导轨上的两个光电门(A、B)、质量为 $m_0$ 的钩码若干、定滑轮。实验装置如图a所示。

\begin{figure}[H]
    \centering
    \subfigure[]{
        \includegraphics[width=0.3\textwidth]{22@4x.png}
    }
    \subfigure[]{
        \includegraphics[width=0.5\textwidth]{2.5cm.jpg}
    }
\end{figure}

(1)实验前,用游标卡尺测量遮光片的宽度$d$,如图b所示(图2中2.5cm的刻度与下方刻度重合),则 $d=$ \uline{\qquad} mm.

(2)为确保测量精度,钩码总质量\uline{\qquad}远大于滑块质量(填“应该”或“无需”或“应该不”)

(3)测的两光电门距离为 $L=45.00$ mm.调整钩码个数,测的多组光电门A的遮光时间 $t_1$和光电门B的遮光时间 $t_2$ .已知当地重力加速度 $g=9.8 \text{m/s}^2$,则滑块M的质量为\uline{\qquad}(用题目所给字母表示)

(4)经统计, $\displaystyle \frac{mt_{1}^{2}t_{2}^{2}}{dt_{1}^{2}-dt_{2}^{2}}$ 的平均值为0.023 kg/m, $m$ 的平均值为0.1kg,则 $M=$ \uline{\qquad} kg

\textbf{23.}一同学用半值法测量一量程为 $ 300 \mu \text{ A}$,内阻约 $100\Omega $ 的电流表的内阻。

(1)实验步骤如下

\begin{enumerate}
    \item 检查电流表是否故障
    \item 闭合 $\rm S_1$,调整 $R_1$ (滑动变阻器)的阻值,是电流表G满偏(或接近满偏),记录 $R_1$ 和G的读数
    \item 闭合 $S_2$ ,调节 $R_1$ ,使其为刚才阻值的一半
    \item 调节 $R_2$ ,使G的指针回到先前记录下的数值
    \item $R_2$ 的阻值即为电流表G的内阻
\end{enumerate}

实验前可利用一根导线检查电流表是否断路,方法是\uline{\qquad}

(2)有以下器材可供选择:

\begin{enumerate}
    \item 待测电流表
    \item 阻值为0-999.9 $\Omega$ 的电阻箱
    \item 阻值为0-9999 $\Omega$ 的电阻箱
    \item 电动势为1.5V,内阻约为 $0.2 \Omega$ 的电源 $E_1$
    \item 电动势为3V,内阻为 $0.5 \Omega$的电源 $E_2$
    \item 开关、导线若干
\end{enumerate}

(3)根据实验步骤画出实验电路图,要求标明 $R_1$、 $R_2$ , $S_1$、 $S_2$ ,使用$E_1$ 还是 $E_2$.

(4)为保护电流表,在 $S_1$ 闭合前, $R_1$应为 \uline{\qquad},$R_2$应为 \uline{\qquad}

\begin{figure}[H]
    \centering

    \subfigure[a]{
        \includegraphics[width=0.2\textwidth]{23a@4x.png}
    }
    \subfigure[b]{
        \includegraphics[width=0.2\textwidth]{23b@4x.png}
    }
    \subfigure[c]{
        \includegraphics[width=0.2\textwidth]{23c@4x.png}
    }
    \subfigure[d]{
        \includegraphics[width=0.2\textwidth]{23d@4x.png}
    }
\end{figure}

(5)根据实验原理,在操作无误的情况下,测量值(\qquad)真实值(填“大于”或“小于”或“等于”)。若测量一个其余条件相同,但内阻比这个表大几十欧姆的电流表,则相对误差(\qquad)(填“明显更大”或“明显更小”或“始终可忽略”)

\begin{wrapfigure}{r}{0.25\textwidth}
    \centering
    \includegraphics[width=0.25\textwidth]{24@4x.png}
\end{wrapfigure}

\textbf{24.}如图,一质量为 $M=5 \text{kg}$ 的装置上放有质量为 $m_1 = 2 \text{kg}$ 的物块,装置的轴上套有质量为 $m_2 = 3 \text{kg}$的物块。装置高 $h=1.7 \text{m}$ 两物块用不可伸长的轻绳连接。装置的上下表面和轴均光滑。装置上表面长 $L=1.2 \texttt{m}$,与轴等长。上表面的物块位于最左端,轴上的物块位于轴顶端时,轻绳恰好绷直。物块大小可忽略。当上表面的物块到达装置最右端时,立刻将轻绳切断。回答下列问题:

(1)若在装置左端对装置施加恒力 $F$,使得装置与两物块共同匀加速直线运动,且三个物体相对静止,求 $F$ 的大小

(2)当上表面的物块落地时,求两物块的距离(结果可用根式表示)

\begin{wrapfigure}{r}{0.25\textwidth}
    \centering
    \includegraphics[width=0.25\textwidth]{25@4x.png}
\end{wrapfigure}

\textbf{25.}如图,绝缘粗糙的竖直平面MN左侧同时存在相互垂直的匀强磁场和匀强电场,电场方向水平向右,大小为$E$,磁感应强度方向垂直纸面向外。磁感应强度大小为$B$。一质量为$m$,电荷量为$q$的带正电荷小滑块从A点开始沿着MN下滑,到达C点时离开MN做曲线运动。AC两点间距离为$h$,重力加速度为$g$


(1)求小滑块运动到C点时的速度$v_c$

(2)求小滑块从A点运动到C点过程中克服摩擦力做的功$W_f$

(3)若DE为小滑块在电场力、洛伦兹力、重力的作用下运动过程中速度最大的位置,当小滑块运动到D点时撤去磁场,此后小滑块继续运动到水平地面上的P点。已知小滑块在D点速度大小为$v_D$,从D点运动到P点的时间为$t$,求小滑块运动到P点的速度大小$v_p$

\textbf{26.}某化学兴趣小组用$\rm Pb$制取$\rm PbCl_2$并用离子交换法测量$\rm PbCl_2$的溶度积。已知$\rm PbCl_2$难溶于冷水,易溶于热水,在$\rm Cl^{-}$浓度高的溶液中生成$\rm [PbCl_4]^{2-}$

I.实验装置如图1(夹持装置省略)

%缺图:装置,柱子

(1)仪器a的名称为\uline{\qquad}

(2)装置需水浴加热至50-60℃,原因是\uline{\qquad \qquad}

(3)反应结束后,\uline{\qquad}(排序,部分操作可重复),得到$\rm PbCl_2$固体

\begin{itemize}
    \item[a] 过滤
    \item[b] 冷却
    \item[c] 洗涤滤渣
    \item[d] 取滤液
\end{itemize}

II.本实验采取强酸型离子交换树脂测定溶度积。交换时所用的树脂为$\rm H^{+}$型树脂,即活性基团为$\rm -SO_3 H$.一定量的$\rm PbCl_2$溶液与$\rm H^+$型阳离子树脂充分接触后,发生反应$\rm 2R-SO_3 H +PbCl_2 =(R-SO_3)_2 Pb +2HCl$,且反应很彻底。

(4)将实验所得$\rm PbCl_2$加热,溶于经煮沸除去$\rm CO_2$并冷却的纯水中,目的是\uline{ \qquad}

(5)取$\rm PbCl_2$饱和溶液100mL,加入准备好的离子交换柱中,应选用\uline{}

\begin{itemize}
    \item[a] 100mL容量瓶
    \item[b] 25mL移液管
    \item[c] 滴定管
\end{itemize}

(6)检验溶液是否已完全通过离子交换柱的方法是\uline{\qquad}

(7)混合所有从离子交换柱中流出的溶液,加入溴百里酚蓝(酸性为黄色,碱性为蓝色)和红色染料(不参与反应),用0.1000mol/L的 $\rm NaOH$溶液滴定。其中加入红色染料的目的是\uline{ \qquad} 。若取25.00mL饱和溶液,滴定消耗9.75mL的$\rm NaOH$标准液,则饱和溶液的浓度为\uline{ \qquad}(按$\rm PbCl_2$记)。于是可得$\rm K_{sp}=$ \uline{ \qquad}(结果保留一位有效数字)

\textbf{27.}$\rm MgSO_4 \cdot 7H_2O$在印染、造纸、医药等工业有广泛应用。$m$是一种无色、无臭、易风化的晶体或白色粉末,易溶于水,在48℃时会失去一个结晶水,在200℃以上会失去所有结晶水 。
盐泥是一种工业废料,含约40\% 的$x \text{MgCO}_3 \cdot y \text{CaCO}_3$,大量的$\rm
    SiO_2$和微量的 $\rm Fe^{3+}$、 $\rm Fe^{2+}$.为充分利用资源,通过如图所示工艺流程从盐泥中提取$m$。已知 $\rm CaSO_4$溶解度随温度上升而降低。

\begin{figure}[H]
    \centering
    \includegraphics[width=\textwidth]{26@4x.png}
\end{figure}

(1)酸浸步骤中,加入硫酸的速度要慢,原因是\uline{ \qquad}

(2)氧化除杂阶段主要是除去铁,氧化时发生的反应的离子方程式是\uline{ \qquad}

(3)图中“?”的操作是\uline{ \qquad},物质A的主要成分是\uline{ \qquad}

(4)根据以下信息,验证“?”操作是否完全除去杂质离子的方法是\uline{ \qquad}

\begin{center}
    \begin{tabular}{c|c | c |c}
        \hline
        物质            & 25℃时的溶解度(g ) & 物质           & 25℃时的溶解度(g ) \\
        \hline
        $\rm CaSO_4$    & 0.205               & $\rm MgCO_3$   & 0.18                \\
        \hline
        $\rm CaC_2 O_4$ & 0.0061              & $\rm Ca(OH)_2$ & 0.160               \\
        \hline
        $\rm MgC_2O_4$  & 0.38                & $\rm Mg(OH)_2$ & 0.00034             \\
        \hline
        $\rm CaCO_3$    & 0.00066                                                    \\
        \hline
    \end{tabular}
\end{center}

(5)为确定 $x$ 和 $y$,取25g盐泥加入足量盐酸,用 $\rm NaClO_3$ 、 $\rm NaOH$ 除去杂质后,得到滤渣13.6g.取0.285g滤液稀释,加入镁试剂(对硝基偶氮间苯二酚)得到蓝色沉淀,用NaOH溶液滴定,终点现象为\uline{\qquad}.消耗12.5mL的NaOH溶液,则 $x=$ \uline{\qquad} , $y=$ \uline{\qquad}

\textbf{28.}焦炭是冶金工业的重要还原剂,许多物质的制备都需要焦炭。回答下列问题。

I.由 $\rm SiO_2$ 制备 $\rm SiCl_4$ 时需要与焦炭共热,发生反应1

\[\rm SiO_2 (s) +2C(s) +2Cl_2 (g) = SiCl_4 (l) +2CO(g)  \qquad \Delta H_1   \]

已知:

反应2

\[\rm SiO_2 (s) +2Cl_2 (g) =SiCl_4 (l) +O_2 (g)  \]

$\rm \Delta H_2= +223.7 kJ/mol$, $\rm \Delta S_2 =+198.7 J\cdot mol^{-1} \cdot K^{-1}$

反应3

\[\rm C(s) + \frac{1}{2} O_2 (g)  =CO (g)\]

$\rm \Delta H_e= -110.5 kJ/mol$, $\rm \Delta S_3 =+192 J\cdot mol^{-1} \cdot K^{-1}$

(1) $\Delta H_1 =$ \uline{\qquad}

(2)在制备$\rm SiCl_4$时添加焦炭的目的:\uline{\qquad}

II.锌锭冶炼是在约1200K下进行的,需要焦炭作还原剂。反应的吉布斯自由能 $\Delta G = \Delta H - T\Delta S$.吉布斯自由能与平衡常数的关系为 $\ln K = -\frac{\Delta G}{RT}$,其中$R=8.314 \text{J/mol} $为普适气体常量

如下吉布斯自由能-温度图中,四条折线表示反应

%缺图:吉布斯自由能

\begin{enumerate}
    \item $\rm Zn+ \frac{1}{2} O_2 = ZnO$
    \item $\rm CO \frac{1}{2}+O_2 =CO_2$
    \item $\rm C+O_2 =CO_2$
    \item $\rm C+\frac{1}{2} O_2 = CO$
\end{enumerate}

(3)其中表示反应1.的折线为\uline{\qquad},原因是\uline{\qquad}

(4)已知焦炭的熔点非常高。折线d在1000K附近时,出现第二个拐点的原因是\uline{\qquad}

(5)1750K时,反应 $\rm 2ZnO +C = CO_2 +Zn$的 $\ln K =$ \uline{\qquad}

(6)最适宜冶炼锌的温度范围为\uline{\qquad},原因是\uline{\qquad}

\textbf{29.}生物

\textbf{30.}生物

\textbf{31.}生物

\textbf{32.}喷瓜是一种很特别的植物,它可以依靠喷射来传播种子。喷瓜没有性染色体,但是有性别。喷瓜的性别由$a^D$、$a^+$、$a^d$三种基因决定,如表所示。

\begin{center}
    \begin{tabular}{c|l}
        \hline
        性别     & 基因型                 \\
        \hline
        雄       & $a^D a^+$或$a^D a^d$   \\
        \hline
        雌雄同株 & $a^+ a^+$ 或 $a^+ a^d$ \\
        \hline
        雌       & $a^d a^d$              \\
        \hline
    \end{tabular}
\end{center}

喷瓜的刚毛长短受一对等位基因(B/b)控制,现发现一短毛雄喷瓜,将其与纯合长毛雌喷瓜杂交,子代表现型为长毛雌雄同株:短毛雄=1:1.不考虑交叉互换,回答下列问题。

(1)喷瓜的性别和刚毛长短\uline{\qquad}自由组合定律(填“遵循”或“不遵循”)

(2)基因B控制的性状为\uline{ \qquad},亲本杂交组合的基因型为\uline{ \qquad}

(3)不考虑性别是否稳定遗传,\uline{ \qquad}利用现有喷瓜得到稳定遗传的短毛喷瓜(填“能”或“不能”)

(4)现有一株短毛雄喷瓜,将其与雌喷瓜杂交,子代雄:雌=1:1,请设计实验验证是否可以利用此瓜和现有喷瓜杂交,得到与亲本基因型完全相同的子代。

\textbf{(二)选考题:共45分。请考生从2道物理题、2道化学题、2道生物题中各科任选一题作答。如果多做,则按每科所做的第一题给分}

\textbf{33.}[物理——选修3-3]

(1)(5分)

(2)(10分)如图,小气缸作为大气缸的活塞。两气缸导热良好、光滑、不漏气。大气缸的底面积为 $S_1=0.25 \text{m}^2$ ,小气缸的底面积为 $S_2=0.2 \text{m}^2$.两气缸长度均为 $L_0 =1 \text{m}$ 环境温度273K,大气压强 $p_0 =101 \text{kPa}$.初始状态大气缸内壁到小气缸外壁距离 $L_1$ ,小气缸内壁到活塞距离 $L_2 =10 \text{cm}$ ,小气缸底部厚度 $L_3 = 3\text{cm}$.

(i)若在保持不漏气的条件下向小气缸内部注入 $V_0=S_2 L_2$ 的液体,求将活塞拉出小气缸所需的拉力 $F$

(ii)拉动活塞,若活塞将要拉出小气缸的时刻,小气缸恰好将要拉出大气缸,求 $L_1$

\textbf{34.}[物理——选修3-4]

\textbf{35.}[化学——选修3:物质结构与性质]

氮及其化合物广泛存在于自然界中,回答下列问题

(1)处于一定空间运动状态的电子在原子核外出现的概率密度分布可用\uline{\qquad}形象化描述。在基态 $\rm ^15 N$中,核外存在\uline{\qquad}种自旋方向相反的电子。

(2) $\rm NO_3$中的氮原子成 $sp^2$ 杂化,其p轨道与氧原子的p轨道共用,形成离域 $\Pi$键。这个离域 $\Pi$键可表示为\uline{\qquad}(例:3原子4电子键可表示为 $\Pi_3^4$)

(3)像 $\rm CN ^-$与 $\rm N_2$这样,具有相同空间构型和键合方式的分子或离子称为等电子体。请再写出一种 $\rm N_2$的等电子体:\uline{\qquad}

(4)$\rm NH_3$的碱性比 $\rm PH_3$ \uline{\qquad}(填“强”或“弱”),原因是\uline{\qquad}

(5)$\rm HNO_2$为弱酸,而$\rm HNO_3$为强酸。$\rm HNO_2$酸性比$\rm HNO_3$强的原因是\uline{\qquad}

(6)$\rm ZnS$在荧光体、光导体材料、涂料、颜料等行业中应用广泛。立方 $\rm ZnS$晶体结构如图所示,其晶胞边长为540.0pm,密度为\uline{\qquad} $\rm g\cdot cm^{-3}$(列式并计算),a位置 $\rm S^{2-}$ 与b位置 $\rm Zn^{2+}$ 的距离为\uline{\qquad}pm(列式表示)

\begin{figure}[H]
    \includegraphics[width=0.2\textwidth]{35.jpg}
\end{figure}

\textbf{36.}[化学——选修5:有机化学基础]

\textbf{37.}[生物——选修1:生物技术实践]

\textbf{38.}[生物——选修3:现代生物科技专题]

\end{document}